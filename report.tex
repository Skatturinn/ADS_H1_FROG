\documentclass[12pt,letterpaper]{article}
\usepackage[utf8]{inputenc}
\usepackage{geometry}
\usepackage{graphicx}
\usepackage{hyperref}
\usepackage{amsmath}
\usepackage{setspace}
\usepackage{titlesec}
\usepackage{lmodern}

\geometry{margin=1in}
\setstretch{1.2}

% Title formatting
\titleformat{\section}{\large\bfseries}{\thesection}{1em}{}
\titleformat{\subsection}{\normalsize\bfseries}{\thesubsection}{1em}{}

% Title page
\title{
	\vspace{2cm}
	\textbf{Project 1: Frog Tail}\\
	\vspace{0.5cm}
	\large Data analysis to understand tail regeneration.\\
	\vspace{1cm}
}
\author{
	Elias Ludviksson \\
	STATGR5243 - Applied Data Science \\
	Columbia University \\
	\vspace{0.5cm}
	\today
}
\date{}

\begin{document}

\maketitle
% \thispagestyle{empty}
\newpage
\begin{abstract}
	% A concise summary of your findings.
	[Write a brief summary of your project, main findings, and conclusions.]
\end{abstract}
\newpage
\setcounter{page}{1}
\tableofcontents
\newpage

\section{Introduction}

Some Xenopus laevis tadpoles have tails which can regenerate after amputation. The tails of these tadpoles contain a variety of cell types, including muscle cells, skin cells, and spinal cord cells.
A study by C.Aztekin et al found the regeneration-organizing cells (ROCs) in the tadpole tails, which are crucial for tail regeneration. They performed single-cell RNA sequencing on these tadpoles to study the gene expression profiles of different cell types during tail regeneration.
The dataset is publicly available \url{https://ftp.ebi.ac.uk/biostudies/fire/E-MTAB-/716/E-MTAB-7716/Files/arrayExpressUpload.zip}[online]. In order to learn the clustering and gene analysis techniques people use on single-cell genomic data. 
We will analyze this dataset using scanpy, scikit, pandas and the numpy libraries in order to find meaning in the dataset.

\section{Methods}
% Detailed description of the data processing and analysis steps
[Describe the data, preprocessing steps, algorithms used, and analysis workflow.]
The data is AnnData formated and contains 13,199 cells × 31,535 genes. 

\subsection{Code Availability}
The code for this project is publicly available at: \\
\url{https://colab.research.google.com/drive/1HZrv7ODnstypcYvlFyD7D5xdXL_oS01C#scrollTo=71a41724}

\section{Results}
% Presentation of the clustering and gene analysis analysis
[Present your findings. Reference two figures: one for clustering results, one for gene expression analysis.]

\begin{figure}[h!]
	\centering
	% \includegraphics[width=0.7\textwidth]{figures/clustering_results.png}
	\caption{Summary of clustering results.}
	\label{fig:clustering}
\end{figure}

\begin{figure}[h!]
	\centering
	% \includegraphics[width=0.7\textwidth]{figures/gene_expression_analysis.png}
	\caption{Gene expression analysis results.}
	\label{fig:gene_expression}
\end{figure}

\section{Conclusion}
% Summary of your findings
[Summarize your main findings, their implications, and possible future work.]

\end{document}